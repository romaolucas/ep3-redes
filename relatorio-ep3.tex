\documentclass[12pt,letterpaper]{article}
\usepackage[brazil]{babel}
\usepackage[utf8]{inputenc}
\usepackage{graphicx}
\usepackage{times}
\usepackage{url}
\usepackage{algorithm}
\usepackage{algorithmic}
\usepackage[bottom=2cm,top=2cm,left=2cm,right=2cm]{geometry}

\title{Relatório do EP3\\MAC0352 -- Redes de Computadores e Sistemas Distribuídos -- 2s2017}
\author{Cesar Cano (8536169) \\ Lucas Romão (8536214) \\ Victor Sprengel (9298002)}
\date{}

\begin{document}
\maketitle

\section{Passo 0}

\textbf{Na definição do protocolo OpenFlow, o que um switch faz toda
vez que ele recebe um pacote que ele nunca recebeu antes?}

O comportamento do switch ao receber um pacote não identificado depende da configuração
das tabelas de fluxo do OpenFlow. O comportamento padrão é enviar o pacote para o controlador
por mensagem através do canal de controle. Uma outra opção é descartar o pacote.

\section{Passo 2}

\textbf{Com o acesso à Internet funcionando em sua rede local, instale
na VM o programa \texttt{traceroute} usando \texttt{sudo apt install
traceroute} e escreva abaixo a saída do comando \texttt{sudo
traceroute -I www.inria.fr}. Pela saída do comando, a partir de qual
salto os pacotes alcançaram um roteador na Europa? Como você chegou a
essa conclusão?} 

\begin{verbatim}
traceroute to www.inria.fr (128.93.162.84), 30 hops max, 60 byte packets
1  192.168.0.1 (192.168.0.1)  3.980 ms  3.835 ms  3.739 ms
2  10.61.0.1 (10.61.0.1)  16.533 ms  21.874 ms  22.779 ms
3  c9062081.virtua.com.br (201.6.32.129)  23.178 ms  23.639 ms  24.326 ms
4  c9062905.virtua.com.br (201.6.41.5)  24.941 ms  30.769 ms  31.640 ms
5  embratel-T0-0-0-1-uacc03.spoph.embratel.net.br (200.212.132.1)  32.390 ms  32.816 ms  33.455 ms
6  ebt-T0-3-0-0-tcore01.spo.embratel.net.br (200.230.159.86)  35.918 ms  18.755 ms  21.817 ms
7  ebt-B11151-intl01.atl.embratel.net.br (200.230.230.32)  145.076 ms  131.500 ms  149.436 ms
8  * * *
9  ae-1-3514.edge2.Atlanta4.Level3.net (4.69.150.197)  138.561 ms  134.929 ms  139.450 ms
10  gtt-level3.Atlanta4.level3.net (4.68.63.158)  137.605 ms  142.098 ms  146.485 ms
11  xe-1-3-1.cr0-par7.ip4.gtt.net (89.149.185.65)  240.778 ms  228.628 ms  245.880 ms
12  renater-gw-ix1.gtt.net (77.67.123.206)  245.475 ms  234.145 ms  239.995 ms
13  193.51.177.107 (193.51.177.107)  231.548 ms  232.486 ms  230.774 ms
14  inria-rocquencourt-te1-4-inria-rtr-021.noc.renater.fr (193.51.184.177)  233.657 ms  237.077 ms  232.875 ms
15  unit240-reth1-vfw-ext-dc1.inria.fr (192.93.122.19)  230.431 ms  229.849 ms  231.854 ms
16  ezp3.inria.fr (128.93.162.84)  229.359 ms  234.286 ms  230.867 ms
\end{verbatim}

Pesquisando pelos endereços IP no site \url{http://geoiplookup.net/}, chegou-se à conclusão de que
os pacotes chegaram a um roteador na Europa no passo 11 onde chegou a um endereço de IP localizado
na Alemanha.

\section{Passo 3 - Parte 1}

\textbf{Execute o comando \texttt{iperf}, conforme descrito no
tutorial, antes de usar a opção \texttt{--switch user}, 5 vezes.
Escreva abaixo o valor médio e o intervalo de confiança da taxa
retornada (considere sempre o primeiro valor do vetor retornado).}

Resultado de 5 execuções do \texttt{iperf}

\begin{verbatim}
mininet> iperf
*** Iperf: testing TCP bandwidth between h1 and h3
*** Results: ['28.6 Gbits/sec', '28.6 Gbits/sec']
mininet> iperf
*** Iperf: testing TCP bandwidth between h1 and h3
*** Results: ['28.1 Gbits/sec', '28.2 Gbits/sec']
mininet> iperf
*** Iperf: testing TCP bandwidth between h1 and h3
*** Results: ['28.1 Gbits/sec', '28.2 Gbits/sec']
mininet> iperf
*** Iperf: testing TCP bandwidth between h1 and h3
*** Results: ['29.4 Gbits/sec', '29.4 Gbits/sec']
mininet> iperf
*** Iperf: testing TCP bandwidth between h1 and h3
*** Results: ['28.5 Gbits/sec', '28.6 Gbits/sec']
\end{verbatim}

Valor médio: 28.54 Gbits/sec
Intervalo de confiança de 95\%: (27.88, 29.20)

\section{Passo 3 - Parte 2}

\textbf{Execute o comando \texttt{iperf}, conforme descrito no
tutorial, com a opção \texttt{--switch user}, 5 vezes. Escreva abaixo
o valor médio e o intervalo de confiança da taxa retornada (considere
sempre o primeiro valor do vetor retornado). O resultado agora
corresponde a quantas vezes menos o da Seção anterior? Qual o motivo
dessa diferença?}

Resultado de mais cinco execuções do comando \texttt{iperf} com a opção
\texttt{--switch-user}:

\begin{verbatim}
mininet> iperf
*** Iperf: testing TCP bandwidth between h1 and h3
*** Results: ['1.28 Mbits/sec', '1.36 Mbits/sec']
mininet> iperf
*** Iperf: testing TCP bandwidth between h1 and h3
*** Results: ['1.45 Mbits/sec', '1.50 Mbits/sec']
mininet> iperf
*** Iperf: testing TCP bandwidth between h1 and h3
*** Results: ['1.44 Mbits/sec', '1.49 Mbits/sec']
mininet> iperf
*** Iperf: testing TCP bandwidth between h1 and h3
*** Results: ['1.28 Mbits/sec', '1.36 Mbits/sec']
mininet> iperf
*** Iperf: testing TCP bandwidth between h1 and h3
*** Results: ['1.44 Mbits/sec', '1.49 Mbits/sec']
\end{verbatim}

Resultado médio: 1.38Mbits/sec
Intervalo de confiança de 95\%: (1.27, 1.49)

O resultado do segundo experimento é 20711 menor em relação ao da seção anterior
isso se deve ao fato de que, quando usada a opção \texttt{--switch-user}, os pacotes
precisam passar entre o espaço de usuário e o espaço do kernel e voltar para o 
espaço de usuário a cada salto ao passo que na seção anterior cada pacote
ficava apenas no espaço de kernel.

\section{Passo 4 - Parte 1}

\textbf{Execute o comando \texttt{iperf}, conforme descrito no
tutorial, usando o controlador \texttt{of\_tutorial.py} original sem
modificação, 5 vezes. Escreva abaixo o valor médio e o intervalo de
confiança da taxa retornada (considere sempre o primeiro valor do
vetor retornado). O resultado agora
corresponde a quantas vezes menos o da Seção 3? Qual o motivo para
essa diferença? Use a saída do comando \texttt{tcpdump}, deixando
claro em quais computadores virtuais ele foi executado, para
justificar a sua resposta.}

Resultado de cinco execuções do comando \texttt{iperf} usando o controlador:
\texttt{of\_tutorial.py}:

\begin{verbatim}
mininet> iperf
*** Iperf: testing TCP bandwidth between h1 and h3
*** Results: ['19.9 Mbits/sec', '22.6 Mbits/sec']
mininet> iperf
*** Iperf: testing TCP bandwidth between h1 and h3
*** Results: ['18.6 Mbits/sec', '21.1 Mbits/sec']
mininet> iperf
*** Iperf: testing TCP bandwidth between h1 and h3
*** Results: ['15.3 Mbits/sec', '16.9 Mbits/sec']
mininet> iperf
*** Iperf: testing TCP bandwidth between h1 and h3
*** Results: ['18.1 Mbits/sec', '20.5 Mbits/sec']
mininet> iperf
*** Iperf: testing TCP bandwidth between h1 and h3
*** Results: ['20.0 Mbits/sec', '24.3 Mbits/sec']
\end{verbatim}

Valor médio: 18.38 Mbits/sec
Intervalo de confiança de 95\%: (16.01, 20.75)

O novo resultado é 1552 vezes menor do que o resultado visto na seção 1. Isso se
deve ao fato de os pacotes irem para o controlador e esse por sua vez manda os pacotes
para todas as portas ao invés de mandar para apenas uma.

\section{Passo 4 - Parte 2}

\textbf{Execute o comando \texttt{iperf}, conforme descrito no
tutorial, usando o seu controlador \texttt{switch.py}, 5 vezes.
Escreva abaixo o valor médio e o intervalo de confiança da taxa
retornada (considere sempre o primeiro valor do vetor retornado). O
resultado agora corresponde a quantas vezes mais o da Seção anterior?
Qual o motivo dessa diferença? Use a saída do comando
\texttt{tcpdump}, deixando claro em quais computadores virtuais ele
foi executado, para justificar a sua resposta.}

\begin{verbatim}
mininet> iperf
*** Iperf: testing TCP bandwidth between h1 and h3
*** Results: ['20.9 Mbits/sec', '24.1 Mbits/sec']
mininet> iperf
*** Iperf: testing TCP bandwidth between h1 and h3
*** Results: ['18.3 Mbits/sec', '20.8 Mbits/sec']
mininet> iperf
*** Iperf: testing TCP bandwidth between h1 and h3
*** Results: ['19.2 Mbits/sec', '21.9 Mbits/sec']
mininet> iperf
*** Iperf: testing TCP bandwidth between h1 and h3
*** Results: ['19.8 Mbits/sec', '22.5 Mbits/sec']
mininet> iperf
*** Iperf: testing TCP bandwidth between h1 and h3
*** Results: ['19.2 Mbits/sec', '22.0 Mbits/sec']
\end{verbatim}

Valor médio: 19.48 Mbits/sec
Intervalo de confiança de 95\%: (18.64, 20.32)

Quase não houve diferença apesar de agora estarmos usando o resend packet quando a porta destino é conhecida,
talvez a diferença entre os computadores que foram usados desse passo pro anterior tenha algo a ver, mas de qualquer forma a mudança não deveria ser muito grande dado que os pacotes ainda são reenviados várias vezes e o switch não cria os fluxos para evitar passar pelo controlador.

\section{Passo 4 - Parte 3}

\textbf{Execute o comando \texttt{iperf}, conforme descrito no
tutorial, usando o seu controlador \texttt{switch.py} melhorado, 5
vezes. Escreva abaixo o valor médio e o intervalo de confiança da taxa
retornada (considere sempre o primeiro valor do vetor retornado). O
resultado agora corresponde a quantas vezes mais o da Seção anterior?
Qual o motivo dessa diferença? Use a saída do comando
\texttt{tcpdump}, deixando claro em quais computadores virtuais ele
foi executado, e saídas do comando \texttt{sudo ovs-ofctl}, com os
devidos parâmetros, para justificar a sua resposta.}

\begin{verbatim}
mininet> iperf 
*** Iperf: testing TCP bandwidth between h1 and h3 
*** Results: ['28.9 Gbits/sec', '28.9 Gbits/sec'] 
mininet> iperf 
*** Iperf: testing TCP bandwidth between h1 and h3 
*** Results: ['28.6 Gbits/sec', '28.7 Gbits/sec'] 
mininet> iperf 
*** Iperf: testing TCP bandwidth between h1 and h3 
*** Results: ['29.9 Gbits/sec', '30.0 Gbits/sec'] 
mininet> iperf 
*** Iperf: testing TCP bandwidth between h1 and h3 
*** Results: ['29.6 Gbits/sec', '29.7 Gbits/sec'] 
mininet> iperf 
*** Iperf: testing TCP bandwidth between h1 and h3 
*** Results: ['27.4 Gbits/sec', '27.5 Gbits/sec']
\end{verbatim}

Valor médio: 28.8 Gbits/sec
Intervalo de confiança de 95\%: (27.94, 29.66)

O resultado é 1571 vezes maior do que o da seção anterior, pois agora os fluxos são adicionados conforme
as portas são descobertas, evitando alguns caminhos que passam pelo controlador após fluxos serem adicionados.

\section{Passo 5}

\textbf{Explique a lógica implementada no seu controlador
\texttt{firewall.py} e mostre saídas de comandos que comprovem que ele
está de fato funcionando (saídas dos comandos \texttt{tcpdump},
\texttt{sudo ovs-ofctl}, \texttt{nc}, \texttt{iperf} e \texttt{telnet}
são recomendadas)}

\section{Configuração dos computadores virtual e real usados nas
medições (se foi usado mais de um, especifique qual passo foi feito
com cada um)}

\section{Referências}

\begin{itemize}
   \item www.comp.nus.edu.sg/~tbma/teaching/cs5229y14\_past/Tutorial.pdf
   \item
   % \item (Coloque quantos itens forem necessários)
\end{itemize}

\end{document}
